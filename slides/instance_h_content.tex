\begin{frame}[fragile, shrink=10]{Header instance.h - Costanti}
  Sono state aggiunte le seguenti costanti nel file instance.h
  %\begin{C}
  \begin{block}{\lstinline[language=C]!#define NUM_AUTORNG_RNG ...!}
  %\end{C}
  Definisce il numero $M$ di misure raccolte durante l'autoranging da ogni ancora master per ogni range di interesse.\\
  \textcolor{dgreen}{esempio:} A0 raccoglie $M$ misure di $r_{01}$, $M$ misure di $r_{02}$ ed $M$ misure di $r_{03}$.
  \end{block}

  \begin{block}{\lstinline[language=C]!#define AUTORANGING_MAX_TIMEOUT_TAG ...!}
  Tempo in \SI{}{\milli\second} dopo il quale il tag inizia la normale procedura di ranging se non
  riceve più messaggi di POLL inviati da un'ancora master.
  \end{block}

  \begin{block}{\lstinline[language=C]!#define  ANCH_FINAL_MSG_LEN 34!}
    Lunghezza del messaggio di tipo \lstinline[language=C]!RTLS_DEMO_MSG_ANCH_FINAL!.\\
    Essa è pari a quella del messaggio di tipo \lstinline[language=C]!RTLS_DEMO_MSG_TAG_FINAL! più uno dato
    che contiene un byte in più che serve ad indicare alle ancore che l'autoranging per la corrente
    ancora master è terminato.
  \end{block}

\end{frame}

\begin{frame}[fragile]{Header instance.h - Costanti}
  \begin{block}{\lstinline[language=C]!#define NUM_ALL_AUTORANING_RANGES!}
  Numero complessivo di range che il tag si aspetta di ricevere da tutte le ancore.\\
  Scritto in funzione del numero totale di ancore $N$.\\
  \textcolor{dgreen}{esempio:} per $4$ ancore
  \[
  |\{r_{01}, r_{02}, r_{03}, r_{12}, r_{13}, r_{23}\}| = 3 + 2 + 1 = 6
  \]
  \alert{in generale:}
  \[
  \frac{(N-1) (N-1+1)}{2} = \frac{(N-1)N}{2}
  \]
  \end{block}
\end{frame}

\begin{frame}[fragile]{Header instance.h - Costanti}
  \begin{block}{\lstinline[language=C]!#define END_AUTORANGING 33!}
    Posizione, all'interno di \lstinline[language=C]!instance_data[0].msg_f.messageData!, del byte
    che indica la fine della procedura di autoranging per una data ancora master.\\
    In questo caso messageData contiene un messaggio di tipo \lstinline[language=C]!RTLS_DEMO_MSG_ANCH_FINAL!.
  \end{block}
  \begin{block}{\lstinline[language=C]!#define AUTORANGING_RANGES 8!}
    Posizione, all'interno di \lstinline[language=C]!instance_data[0].msg_f.messageData!, a partire dalla quale
    l'ancora inserisce le medie dei range che le competono.\\
    In questo caso messageData contiene un messaggio di tipo \lstinline[language=C]!RTLS_DEMO_MSG_ANCH_RESP!.
  \end{block}
\end{frame}

\begin{frame}[fragile]{Header instance.h - Struct \lstinline[language=C]!sfConfig_t!}
  Le seguenti \lstinline[language=C]!struct! sono state modificate
  \begin{C}
    typedef struct
    {
      ...
      uint16 tagPollSleepDly;
      uint16 anchPollSleepDly;
      ...
    } sfConfig_t;
  \end{C}
  È stato aggiunto il campo \lstinline[language=C]!anchPollSleepDly! e il campo \lstinline[language=C]!PollSleepDly! è stato rinominato
  TagPollSleepDly per maggiore chiarezza.\\
  Il campo \lstinline[language=C]!anchPollSleepDly! rappresenta il periodo in \SI{}{\milli\second} di trasmissione dei Poll
  dell'ancora master.

\end{frame}

\begin{frame}[fragile]{Header instance.h - Struct \lstinline[language=C]!instance_data_t!}
  \begin{C}
    typedef struct
    {
      ...
    } instance_data_t;
  \end{C}
  Sono stati aggiunti i seguenti campi
  \begin{block}{\lstinline[language=C]!int32 anchSleepTime_ms!}
    Periodo in \SI{}{\milli\second} di trasmissione dei Poll dell'ancora master
  \end{block}
  
  \begin{block}{\lstinline[language=C]!unsigned long autoranging_poll_time!}
    Se il dispositivo si comporta come ancora in modalità \lstinline[language=C]!ANCHOR_RNG! ed è l'ancora master
    rappresenta l'istante di tempo, secondo l'orologio interno, in cui è stato spedito l'ultimo Poll.\\
    Utilizzato per capire se sono passati \lstinline[language=C]!anchSleepTime_ms! \SI{}{\milli\second} dall'invio dell'ultimo Poll.\\
  \end{block}
\end{frame}

\begin{frame}[fragile]{Header instance.h - Struct \lstinline[language=C]!instance_data_t!}
  \begin{block}{}
    Se invece il dispositivo si comporta come tag in modalità \lstinline[language=C]!TAG_WAIT! rappresenta l'istante di tempo, secondo l'orologio
    interno, in cui il tag ha ricevuto l'ultimo Poll inviato da un'ancora master.\\
    Utilizzato per capire se sono passati \lstinline[language=C]!AUTORANGING_MAX_TIMEOUT_TAG! \SI{}{\milli\second} dalla ricezione
    dell'ultimo Poll inviato da un'ancora master.
  \end{block}
  \begin{block}{\lstinline[language=C]!uint8 autoranging_timeout!}
    Abilitazione del timer usato da un'ancora master per inviare periodicamente Poll alle altre ancore.\\
    Abilitazione del timer usato da un tag in modalità \lstinline[language=C]!TAG_WAIT! per attendere la fine della
    procedura di autoranging.
  \end{block}
\end{frame}

\begin{frame}[fragile]{Header instance.h - Struct \lstinline[language=C]!instance_data_t!}
  \begin{block}{\lstinline[language=C]!double anchRngArray[MAX_ANCHOR_LIST_SIZE]!}
    Contiene la somma dei range raccolti da una data ancora.\\
    \textcolor{dgreen}{esempio:} sia l'ancora A1. \lstinline[language=C]!anchrRngArray[0]! contiene la somma dei range $r_{01}$,
    \lstinline[language=C]!anchrRngArray[2]! contiene la somma dei range $r_{12}$ e \lstinline[language=C]!anchrRngArray[3]! contiene la somma dei range $r_{13}$.
  \end{block}
  \begin{block}{\lstinline[language=C]!uint16 anchRngArrayCounter!}
    Contiene il numero di misure raccolte per ciascun range di interesse per una data ancora.\\
    \textcolor{dgreen}{esempio:} sia l'ancora A1. \lstinline[language=C]!anchrRngArrayCounter[0]! contiene il numero di misure del range $r_{01}$,
    \lstinline[language=C]!anchrRngArrayCounter[2]! contiene il numero di misure del range $r_{12}$ e \lstinline[language=C]!anchrRngArrayCounter[3]! contiene il numero di misure del range $r_{13}$.
  \end{block}
\end{frame}

\begin{frame}[fragile]{Header instance.h - Struct \lstinline[language=C]!instance_data_t!}
  \begin{block}{\lstinline[language=C]!uint8 autoRngRangesRxMask!}
    Bitmask. Il bit i-esimo vale 1 se il tag ha ricevuto la medie dei range dall'ancora i-esima.
  \end{block}
  \begin{block}{\lstinline[language=C]!double autoRngRangesArray[NUM_ALL_AUTORANGING_RANGES]!}
    Medie dei range ricevute dal tag.\\
    \textcolor{dgreen}{esempio:} Nel caso di 4 ancore \lstinline[language=C]!autoRngRangesArray[0]! $ = r_{01}$,
    \hdots, \lstinline[language=C]!autoRngRangesArray[5]! $ = r_{23}$.
  \end{block}
  \begin{block}{\lstinline[language=C]!float anchorPositionMatrix[3 * MAX_ANCHOR_LIST_SIZE]!}
    Posizioni cartesiane delle ancore ottenute con apposito algoritmo a partire dalle medie dei range salvate
    in \lstinline[language=C]!autoRngRangesArray!.\\
    \alert{in generale:} \lstinline[language=C]!anchorPositionMatrix[i][j]! contiene la coordinata $i$-esima dell'ancora $j$-esima.
  \end{block}
\end{frame}

\begin{frame}
  \begin{block}{\lstinline[language=C]!uint8 tagPositionsSentToViewer!}
    Contatore utilizzato per inviare ad intervalli regolari la posizione delle ancore sulla porta seriale
    virtuale (USB OTG VCP). Tali posizioni sono utilizzate dal Viewer 3D.
  \end{block}
  \begin{block}{\lstinline[language=C]!uint16 anchorRngMaster!}
    Contiene l'ID corrente dell'ancora master durante l'autoranging.
  \end{block}
\end{frame}

\begin{frame}[fragile]{Header instance.h - Enum}
  Il seguente enum è stato modificato
  \begin{C}
    typedef enum instanceModes{LISTENER, TAG, TAG_WAIT, ANCHOR, ANCHOR_RNG, NUM_MODES} INST_MODE;
  \end{C}
  È stato aggiunto l'enumeratore \lstinline[language=C]!TAG_WAIT!. Esso corrisponde alla modalità in cui il tag
  attende il completamento della procedura di autoranging.
\end{frame}


