\begin{frame}[fragile, shrink=15]{File main.c}
  Nel seguito sono riportate alcune note riguardanti il file \lstinline!main.c!
  \begin{block}{Configurazione Slot and Superframe}
    In generale è stato aggiunto il campo relativo al periodo di trasmissione dei Poll
    da parte di un'ancora master. È stato verificato sperimentalmente che un periodo di \SI{10}{\milli\second} è
    sufficiente a gestire 4 ancore di cui una master.\\
    \begin{lstlisting}
      sfConfig_t sfConfig[4] =
      {
      ...
      //mode 2 - S1: 2 on, 3 off
      {
        (10),   // slot period (ms)
        (1),    // number of slots
        (10*1), // superframe period (ms)
        (10*1), // poll sleep delay (ms)
        (10),   // anch sleep delay in autoranging (ms)
        (2500)  // final transmission delay
      },
      ...
    };
  \end{lstlisting}
  \end{block}
\end{frame}

\begin{frame}[fragile, shrink=15]{File main.c}
  \begin{block}{\lstinline!uint32 inittestapplication(uint8 s1switch)!}
    La modalità di funzionamento di un dispositivo DecaWaveEVB1000 dipende
    dallo switch $n.4$ presente sul PCB.\\
    Un dispositivo di tipo tag, nella nuova release, non viene più inizializzato in modalità
    \lstinline!TAG! ma \lstinline!TAG_WAIT!. In tale modalità attende il termine della procedura
    di autoranging.\\
    Un dispositivo di tipo ancora, nella nuova release, non viene più inizializzato in modalità
    \lstinline!ANCHOR! ma \lstinline!ANCHOR_RNG!. In tale modalità le ancore eseguono la procedura
    di autoranging.
    \begin{lstlisting}
    if((s1switch & SWS1_ANC_MODE) == 0)
    {
      instance_mode = TAG_WAIT;
    }
    else
    {
      instance_mode = ANCHOR_RNG;
    }
    \end{lstlisting}
  \end{block}
\end{frame}

\begin{frame}[fragile, shrink=15]{File main.c}
  \begin{block}{\lstinline!void setLCDline1(uint8 s1switch)!}
    Questa funzione gestisce il testo riportato sul display LCD a seconda
    della modalità del dispositivo DecaWaveEVB1000.\\
    Nella nuova release del firmware i tag all'accensione riportano la scritta
    ``Waiting...'' ad indicare che attende il termine della procedura di autoranging.
    Le ancore invece riportano la scritta ``Autoranging''.
    \begin{lstlisting}
      ...
      else if(role == TAG_WAIT)
      {
        sprintf((char*)&dataseq1[0], "Tag:%d Waiting...", tagaddr);
        writetoLCD( 16, 1, dataseq1);
      }
      ...
      else if(role == ANCHOR_RNG)
      {
        sprintf((char*)&dataseq1[0], "A:%d Autoranging", ancaddr);
        writetoLCD( 16, 1, dataseq1);
      }
      ...
    \end{lstlisting}
  \end{block}
\end{frame}

\begin{frame}[fragile, shrink=30]{File main.c}
  \begin{block}{\lstinline!int main(void)!}
    Rispetto alla release precedente, il tag, dopo aver ricevuto i tof da tutte le ancore,
    invia il risulutato della trilaterazione attraverso la porta seriale (USB OTG VCP)
    \begin{lstlisting}
    ...  
    if(rx == TOF_REPORT_T2A)
    {
      if (instance_data[0].mode == TAG)
      {
        ...
        instance_data[0].tagPositionSentToViewer++;

  	 float pos_x = (float) best_solution.x;
	 float pos_y = (float) best_solution.y;
	 float pos_z = (float) best_solution.z;
	 print_over_otg_usb("tpr %d %08x %08x %08x",
                            (instance_data[0].instanceAddress16 & 0x7),
	                    *(uint32_t*)&pos_x,
                            *(uint32_t*)&pos_y,
                            *(uint32_t*)&pos_z);
    \end{lstlisting}
  \end{block}
\end{frame}

\begin{frame}[fragile, shrink=30]{File main.c}
  \begin{block}{\lstinline!int main(void)! - continuazione}
    Inoltre ogni \lstinline!SEND_ANCHOR_POSITION_EVERY_CYCLES! cicli invia sulla medesima
    porta la posizione cartesiana delle ancore.
    \begin{lstlisting}
	  if(instance_data[0].tagPositionSentToViewer == SEND_ANCHOR_POSITION_EVERY_CYCLES)
	  {
	    instance_data[0].tagPositionSentToViewer = 0;
	    print_over_otg_usb("apr %d %08x %08x %08x %08x %08x %08x %08x %08x %08x %08x %08x %08x",
			       (instance_data[0].instanceAddress16 & 0x7),
	                        *(uint32_t*)&(instance_data[0].
                                anchorPositionMatrix[0][0]),
			        ...
			        *(uint32_t*)&(instance_data[0].
                                anchorPositionMatrix[2][3]));
	  }
          ...
        }
        \end{lstlisting}
  \end{block}
\end{frame}
