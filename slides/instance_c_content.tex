\begin{frame}[fragile]{File instance.c - \lstinline!testapprun!}
  La funzione
  \begin{lstlisting}
    int testapprun(instance_data_t *inst, int message)
  \end{lstlisting}
  contiene una macchina a stati che implementa la logica di funzionamento delle ancore e dei tag.
  Essa è stata modificata al fine di rimuovere la vecchia modalità di autoranging ed inserire la nuova.\\
  Nel seguito sono descritti i principali \alert{cambiamenti} apportati.
\end{frame}

\begin{frame}[fragile]{File instance.c - \lstinline!testapprun!}
  \begin{block}{Stato \lstinline!TA_INIT!}
    Un tag in modalità \lstinline!TAG_WAIT! abilita la ricezione di eventuali messaggi
    provenienti dalle ancore che eseguono l'autoranging.\\
    Il codice
    \begin{lstlisting}
      inst->autoranging_timeout = TRUE;
      inst->autoranging_poll_time = portGetTickCount();
    \end{lstlisting}
    abilita il timeout software utilizzato per attendere la fine della procedura di autoranging e
    memorizza l'istante di tempo corrente nel caso in cui la procedura di autoranging fosse terminata
    prima dell'accensione del tag.
  \end{block}
\end{frame}

\begin{frame}[fragile]{File instance.c - \lstinline!testapprun!}
  \begin{block}{Stato \lstinline!TA_INIT! - continuazione}
    Di solito (vedi ??), infatti, \lstinline!autoranging_poll_time! viene assegnato
    alla ricezione da parte del tag di un Poll inviato da un'ancora master.\\
    Il codice
    \begin{lstlisting}
      inst->autoRngRangesRxMask = 0;
    \end{lstlisting}
    inizializza la maschera che indica da quali ancore il tag ha ricevuto i range medi.\\
    \alert{Lo stato successivo} è \lstinline!TA_RXE_WAIT!.
  \end{block}
\end{frame}

\begin{frame}[fragile]{File instance.c - \lstinline!testapprun!}
  \begin{block}{Stato \lstinline!TA_INIT! - continuazione}
    Un'ancora in modalità \lstinline!ANCHOR_RNG! si comporta diversamente a seconda che sia l'ancora master o meno.\\
    L'ancora A0 viene configurata come ancora master e preparata per eseguire il primo sleep prima dell'invio del primo Poll.
    Il codice
    \begin{lstlisting}
      for(i=0; i<MAX_ANCHOR_LIST_SIZE; i++)
          inst->anchRngArrayCounter[i] = 0;
      
      uint16 instance_address = inst->instanceAddress16 & 0x3;
      inst->anchRngArrayCounter[instance_address] = 0xFFFF - 0x1;
    \end{lstlisting}
    si occupa di inizializzare il contatore delle misure raccolte per ogni range di interesse.
  \end{block}
\end{frame}


\begin{frame}[fragile]{File instance.c - \lstinline!testapprun!}
  \begin{block}{Stato \lstinline!TA_INIT! - continuazione}
    Per ogni ancora master, sia l'$i$-esima, $i \ge 0$, la procedura di autoranging si considera terminata
    quando l'elemento più piccolo dell'array \lstinline!anchRngArrayCounter! è maggiore o uguale a
    \lstinline!NUM_AUTORNG_RGN!. Dal momento che l'elemento $i$-esimo dell'array, che si riferisce al range
    $r_{ii}$, non viene mai aggiornato esso viene posto al massimo numero rappresentabile su $16$ bit.\\
    \alert{Lo stato successivo} è \lstinline!TA_TXE_WAIT!.
  \end{block}
\end{frame}

\begin{frame}[fragile]{File instance.c - \lstinline!testapprun!}
  \begin{block}{Stato \lstinline!TA_INIT! - continuazione}
    Le altre ancore, con indirizzo diverso da A0, abilitano la ricezione
    e rimangono in attesa di messaggi di Poll inviati dall'ancora master.\\
    \alert{Lo stato successivo} è \lstinline!TA_RXE_WAIT!.
  \end{block}
\end{frame}

\begin{frame}[fragile, shrink=10]{File instance.c - \lstinline!testapprun!}
  \begin{block}{Stato \lstinline!TA_SLEEP_DONE!}
    In questo stato i tag, nel loro funzionamento normale, e le ancore master, in modalità
    \lstinline!ANCHOR_RNG!, attendono che sia il momento di inviare un nuovo Poll.\\
    Per le ancore è stato deciso di non utilizzare la modalità \lstinline!DEEP_SLEEP!
    per cui, pur entrando nel ramo \lstinline!DEEP_SLEEP == 1!, viene replicata l'istruzione
    \lstinline!Sleep(3)! prevista nel ramo \lstinline!#else!.
    \begin{lstlisting}
      #if (DEEP_SLEEP == 1)
          if (inst->mode == TAG)
          {
            ...
          }  
          else if (inst->mode == ANCHOR_RNG)
              Sleep(3);
      #else
          Sleep(3);
      #endif
    \end{lstlisting}
  \end{block}
\end{frame}

\begin{frame}[fragile, shrink=10]{File instance.c - \lstinline!testapprun!}
  \begin{block}{Stato \lstinline!TA_TXE_WAIT!}
    In questo stato i tag, nel loro funzionamento normale, e le ancore master, in modalità
    \lstinline!ANCHOR_RNG!, eseguono alcune operazioni in preparazione alla spedizione
    del successivo Poll.\\
    In particolare con le istruzioni
    \begin{lstlisting}
      if (inst->mode == ANCHOR_RNG)
          inst->rangeNumAnc++;
      ...
      else if (inst->mode == ANCHOR_RNG)
          inst->rxResponseMaskAnc = 0;
    \end{lstlisting}
    l'ancora master, replicando il comportamento di un tag, incrementa il rangeNumber,
    utilizzato nel protocollo di trasmissione per eseguire un integrity check, e riporta
    a zero la maschera che indica quali ancore hanno risposto al Poll inviato.
  \end{block}
\end{frame}

\begin{frame}[fragile, shrink=10]{File instance.c - \lstinline!testapprun!}
  \begin{block}{Stato \lstinline!TA_TX_POLL_WAIT_SEND!}
    In questo stato i tag, nel loro funzionamento normale, e le ancore master, in modalità
    \lstinline!ANCHOR_RNG!, spediscono effettivamente il Poll.\\
    In particolare con le istruzioni
    \begin{lstlisting}
      inst->responseTO = MAX_ANCHOR_LIST_SIZE - 1;
      dwt_setrxtimeout((uint16)inst->fwtoTime_sy * (MAX_ANCHOR_LIST_SIZE - 1));
    \end{lstlisting}
    l'ancora master, replicando il comportamento di un tag, configura un timeout di ricezione
    di durata proporzionale al numero di risposte attese \lstinline!MAX_ANCHOR_LIST_SIZE - 1!.
    Tale numero è inferiore di uno rispetto al numero di risposte attese normalmente da un tag
    dato che una delle ancore si comporta come master.
  \end{block}
\end{frame}

\begin{frame}[fragile]{File instance.c - \lstinline!testapprun!}
  \begin{block}{Stato \lstinline!TA_TX_FINAL_WAIT_SEND!}
    In questo stato i tag, nel loro funzionamento normale, e le ancore master, in modalità
    \lstinline!ANCHOR_RNG!, spediscono effettivamente il Final.\\
    Dato che l'ancora master invia un byte in più per specificare se la \alert{propria} procedura di autoranging
    è terminata è necessario modificare la lunghezza del messaggio con l'istruzione
    \begin{lstlisting}
      inst->psduLength = (ANCH_FINAL_MSG_LEN + FRAME_CRTL_AND_ADDRESS_S + FRAME_CRC);
    \end{lstlisting}
    dove \lstinline!ANCH_FINAL_MSG_LEN! vale $34$ invece che $33$ come nel caso del tag.\\
  \end{block}
\end{frame}

\begin{frame}[fragile]{File instance.c - \lstinline!testapprun!}
  \begin{block}{Stato \lstinline!TA_TX_FINAL_WAIT_SEND! - continuazione}
    Con le istruzioni
    \begin{lstlisting}
      inst->msg_f.messageData[END_AUTORANGING] = FALSE;
      ...
      if(inst->mode == ANCHOR_RNG && 
      array_min(inst->anchRngArrayCounter, MAX_ANCHOR_LIST_SIZE) >= NUM_AUTORNG_RNG)
      {
        inst->anchorRngMaster++;
        inst->msg_f.messageData[END_AUTORANGING] = TRUE;
      }
    \end{lstlisting}
    viene modificato il flag che indica se la procedura di autoranging è terminata e viene
    incrementata la variabile contenente l'ID dell'ancora master corrente.
  \end{block}
\end{frame}

\begin{frame}[fragile, shrink=30]{File instance.c - \lstinline!testapprun!}
  \begin{block}{Stato \lstinline!TA_TX_FINAL_WAIT_CONF!}
    In questo stato i tag, nel loro funzionamento normale, e le ancore master, in modalità
    \lstinline!ANCHOR_RNG!, verificano l'avvenuta spedizione del Poll e del Final.\\
    Per quanto riguarda l'ancora master, nel caso in cui stia verificando l'avvenuta spedizione
    di un Poll, si comporta come un tag.\\
    Se invece l'ancora master sta verificando la spedizione di un Final e viene soddisfatto il criterio
    di termine della procedura di autoranging allora si porta nello stato di ricezione
    \lstinline!TA_RXE_WAIT! in cui attenderà la ricezione di nuovi Poll provenienti dalla nuova ancora master.
    \begin{lstlisting}
      if(inst->previousState == TA_TXFINAL_WAIT_SEND)
      {
        if(inst->mode == ANCHOR_RNG && 
        array_min(inst->anchRngArrayCounter, MAX_ANCHOR_LIST_SIZE) >= NUM_AUTORNG_RNG)
        {
          ...
          inst->testAppState = TA_RXE_WAIT;
          ...
    \end{lstlisting}
  \end{block}
\end{frame}

\begin{frame}[fragile, shrink=30]{File instance.c - \lstinline!testapprun!}
  \begin{block}{Stato \lstinline!TA_TX_FINAL_WAIT_CONF! - continuazione}
    Inoltre se, dopo aver incrementato il contatore \lstinline!anchorRngMaster!,
    l'indirizzo della nuova ancora master è pari
    al numero totale di ancore allora la procedura \alert{complessiva} di autoranging
    è terminata ed è possibile ripristinare la modalità normale di funzionamento \lstinline!ANCHOR!
    attraverso la chiamata alla funzione \lstinline!instance_backtoanchor()!.
    \begin{lstlisting}
          if((inst->anchorRngMaster & 0x7) == MAX_ANCHOR_LIST_SIZE))
            instance_backtoanchor(inst);
          ...
          }
    \end{lstlisting}
    \alert{Solo l'ultima ancora} chiama la funzione \lstinline!instance_backtoanchor()!
    in questa parte della macchina a stati. Le altre ancore rilevano il termine della procedura
    di autoranging quando ricevono l'ultimo final inviato dall'ultima
    ancora nello stato \lstinline!TA_RX_WAIT_DATA!.
  \end{block}
\end{frame}

\begin{frame}{File instance.c - \lstinline!testapprun! - \lstinline!TA_RX_WAIT_DATA!}
  Lo stato \lstinline!TA_RX_WAIT_DATA!, assieme alla callback di ricezione \lstinline!instance_rxcallback()!
  contenuta nel file \lstinline!instance_common.c!, si occupa di processare i messaggi ricevuti dal dispositivo,
  ancora o tag, e ne modifica il comportamento di consenguenza.\\
  Nel caso in cui il dispositivo abbia ricevuto un messaggio valido, lo stato \lstinline!TA_RX_WAIT_DATA!
  prevede comportamenti diversi in base all'\lstinline!fcode! contenuto nel messaggio.\\
  Di seguito sono descritti i principali \alert{cambiamenti} apportati a questo stato suddivisi
  per \lstinline!fcode!.
\end{frame}

\begin{frame}[fragile, shrink=30]{File instance.c - \lstinline!testapprun! - \lstinline!TA_RX_WAIT_DATA!}
  Il firmware originale gestiva gli fcode \lstinline!RTLS_DEMO_MSG_TAG_POLL! e
  \lstinline!RTLS_DEMO_MSG_ANCH_POLL! assieme. Nella nuova release sono gestiti separatamente.
  \begin{block}{fcode \lstinline!RTLS_DEMO_MSG_ANCH_POLL!}
    Le ancore che ricevono un Poll da parte di un'ancora master si comportano come le ancore
    che ricevono un Poll da parte di un tag nella normale procedura di ranging.\\
    Un tag nella modalità \lstinline!TAG_WAIT! reimposta il tempo di ricezione del Poll in modo
    da garantire il funzionamento del sistema di timeout con cui il tag si rende conto del termine
    della procedura di autoranging. In questa situazione il tag torna ad ascoltare nello stato
    \lstinline!TA_RXE_WAIT!.
    \begin{lstlisting}
      ...
      if (inst->mode == TAG_WAIT)
      {
        inst->autoranging_poll_time = portGetTickCount();
        inst->testAppState = TA_RXE_WAIT;

        break;
      }
      ...
    \end{lstlisting}
  \end{block}
\end{frame}

\begin{frame}[fragile]{File instance.c - \lstinline!testapprun! - \lstinline!TA_RX_WAIT_DATA!}
  Il firmware originale gestiva gli fcode \lstinline!RTLS_DEMO_MSG_ANCH_RESP! e \lstinline!_RESP2!
  assieme. Nella nuova release sono gestiti separatamente.
  \begin{block}{fcode \lstinline!RTLS_DEMO_MSG_ANCH_RESP!}
    Il tag che riceve una risposta da parte di un'ancora si occupa anche di gestire i range medi
    elaborati durante la fase di autoranging e che ogni ancora si occupa di inviare al tag in risposta
    ad un Poll (oltre al Tof come previsto nel funzionamento normale della procedura di ranging).\\
    In particolare se il tag non ha ancora ricevuto i range medi dall'ancora avente indirizzo
    \lstinline!srcAddr[0]!
    \begin{lstlisting}
      if ((inst->mode == TAG) && (inst->autoRngRangesRxMask & (0x1 << (srcAddr[0]&0x3))) == 0)
      {
    \end{lstlisting}
  \end{block}
\end{frame}

\begin{frame}[fragile]{File instance.c - \lstinline!testapprun! - \lstinline!TA_RX_WAIT_DATA!}
  \begin{block}{fcode \lstinline!RTLS_DEMO_MSG_ANCH_RESP! - continuazione}
    valuta il numero di range che si aspetta di ricevere da tale ancora
    \begin{lstlisting}
        uint8 number_of_ranges = MAX_ANCHOR_LIST_SIZE - 1 - (srcAddr[0]&0x3);
    \end{lstlisting}
    e copia i dati dal messaggio appena ricevuto nella struttura dati \lstinline!autoRngRangesArray!
    a partire dal byte \lstinline!offset!-esimo.
    \begin{lstlisting}
        uint8 offset = 0;
        for(i = 0; i < (srcAddr[0]&0x3); i++)
          offset += (MAX_ANCHOR_LIST_SIZE - 1 - i);

        memcpy(&(inst->autoRngRangesArray[offset]), &(messageData[AUTORANGING_RANGES]),
        sizeof(double) * number_of_ranges);
    \end{lstlisting}
  \end{block}
\end{frame}

\begin{frame}[fragile, shrink=15]{File instance.c - \lstinline!testapprun! - \lstinline!TA_RX_WAIT_DATA!}
  \begin{block}{fcode \lstinline!RTLS_DEMO_MSG_ANCH_RESP! - continuazione}
    Inoltre aggiorna la maschera \lstinline!autoRngRangesRxMask! in modo da non ripetere
    la procedura di salvataggio dei range medi alla ricezione di altre risposte dalla medesima ancora.
    \begin{lstlisting}
        inst->autoRngRangesRxMask |= (0x1 << (srcAddr[0]&0x3));
    \end{lstlisting}
    Se il tag ha ricevuto tutti i range medi da tutte le ancora valuta le posizioni cartesiane
    delle ancore attraverso la chiamata alla funzione \lstinline!rangeToPos! che salva il risultato
    nella matrice \lstinline!anchorPositionMatrix!.
    \begin{lstlisting}
        if (inst->autoRngRangesRxMask  == (0x1 << MAX_ANCHOR_LIST_SIZE) - 1)
        {
          rangesToPos(inst->autoRngRangesArray,\
          inst->anchorPositionMatrix);
        }
      }
      ...
    \end{lstlisting}
  \end{block}
\end{frame}

\begin{frame}[fragile, shrink=15]{File instance.c - \lstinline!testapprun! - \lstinline!TA_RX_WAIT_DATA!}
  Il firmware originale gestiva gli fcode \lstinline!RTLS_DEMO_MSG_ANCH_RESP! e
  \lstinline!RTLS_DEMO_MSG_ANCH_RESP2! assieme. Nella nuova release sono gestiti separatamente.
  \begin{block}{fcode \lstinline!RTLS_DEMO_MSG_ANCH_RESP2!}
    Un tag in \lstinline!TAG_WAIT! ignora il messaggio e riabilita la ricezione.\\
    Le ancore, master o meno, si comportano come in una normale procedura di ranging
    eccetto per il fatto che l'ancora master salva il tof ricevuto in un accumulatore
    \lstinline!anchRngArray! e incrementa il contatore \lstinline!anchRngArrayCounter!.
    \begin{lstlisting}
      ...
      if (inst->anchRngArrayCounter[src_addr] == 0)
        inst->anchRngArray[src_addr] = range;
      else 
        inst->anchRngArray[src_addr] += range;
      inst->anchRngArrayCounter[src_addr]++;
      ...
    \end{lstlisting}
  \end{block}
\end{frame}

\begin{frame}[fragile]{File instance.c - \lstinline!testapprun! - \lstinline!TA_RX_WAIT_DATA!}
  \begin{block}{fcode \lstinline!RTLS_DEMO_MSG_TAG_FINAL! e \lstinline!RTLS_DEMO_MSG_ANCH_FINAL!}
    Un tag in \lstinline!TAG_WAIT! e riabilita la ricezione.\\
    Un'ancora non master che riceve un \lstinline!ANCH_FINAL! calcola il tof, come
    farebbe nel caso in cui ha ricevuto un \lstinline!TAG_FINAL!, lo salva in un accumulatore
    \lstinline!anchRngArray! ed incrementa il contatore \lstinline!anchRngArrayCounter!
    \begin{lstlisting}
      if (inst->anchRngArrayCounter[src_addr] == 0)
        inst->anchRngArray[src_addr] = range;
      else 
        inst->anchRngArray[src_addr] += range;
      inst->anchRngArrayCounter[src_addr]++;
    \end{lstlisting}
  \end{block}
\end{frame}

\begin{frame}[fragile]{File instance.c - \lstinline!testapprun! - \lstinline!TA_RX_WAIT_DATA!}
  \begin{block}{fcode \lstinline!RTLS_DEMO_MSG_ANCH_FINAL! - continuazione}
    Inoltre l'ancora verifica se il messaggio di Final ricevuto contiene
    il byte che indica la fine della procedura di autoranging da parte dell'ancora
    master corrente. Nel caso in cui la procedura sia terminata viene incrementato
    l'indirizzo dell'ancora master corrente. Inoltre \alert{viene invalidato} il tof
    dal momento che non potrebbe essere utilizzato nella risposta ad un Poll proveniente
    dalla successiva ancora master dato che è riferito all'ancora master precedente.
    \begin{lstlisting}
      if(messageData[END_AUTORANGING] == TRUE)
      {
        inst->anchorRngMaster++;
        inst->autoranging_timeout = FALSE;
        inst->tofAnc = INVALID_TOF;
      }
    \end{lstlisting}
  \end{block}
\end{frame}

\begin{frame}[fragile, shrink=30]{File instance.c - \lstinline!testapprun! - \lstinline!TA_RX_WAIT_DATA!}
  \begin{block}{fcode \lstinline!RTLS_DEMO_MSG_ANCH_FINAL! - continuazione}
    Qualora l'ancora che ha ricevuto il final fosse la nuova ancora master
    si prepara per spedire il suo primo Poll (le inizializzazioni sono le medesime
    eseguite dall'ancora 0 nello stato \lstinline!TA_INIT!).
    \begin{lstlisting}
      if(inst->mode == ANCHOR_RNG && inst->instanceAddress16 == inst->anchorRngMaster)
      {
        ...
        inst->autoranging_timeout = FALSE;
        inst->nextState = TA_TXPOLL_WAIT_SEND;
        inst->testAppState = TA_TXE_WAIT;
        
        inst->rangeNumAnc = 0;

        for(i=0; i<MAX_ANCHOR_LIST_SIZE; i++)
            inst->anchRngArrayCounter[i] = 0;

        uint16 instance_address = inst->instanceAddress16 & 0x3;
        inst->anchRngArrayCounter[instance_address] = 0xFFFF - 0x1;
        ...
      }
    \end{lstlisting}
  \end{block}
\end{frame}

\begin{frame}[fragile]{File instance.c - \lstinline!testapprun! - \lstinline!TA_RX_WAIT_DATA!}
  \begin{block}{fcode \lstinline!RTLS_DEMO_MSG_ANCH_FINAL! - continuazione}
    Se al contrario l'ancora rileva che l'intera procedura di autoranging è completata,
    poichè l'indirizzo della nuova ancora master è pari al numero totale di ancore, calcola le medie
    dei range attraverso la funzione \lstinline!eval_means()! e ripristina la modalità normale di funzionamento
    \lstinline!ANCHOR! attraverso la funzione \lstinline!instance_backtoanchor()!.
    \begin{lstlisting}
      ...
      else if((inst->mode == ANCHOR_RNG) && (inst->anchorRngMaster & 0x7) == MAX_ANCHOR_LIST_SIZE)
      {
        eval_means(inst);
        instance_backtoanchor(inst);
      }
      ...
    \end{lstlisting}
  \end{block}
\end{frame}

\begin{frame}[fragile]{File instance.c - \lstinline!array_min!}
  La seguente funzione è stata aggiunta
  \begin{block}{\lstinline!uint16 array_min (uint16* array, int len)!}
    Calcola il minimo di un array.
  \end{block}
\end{frame}
